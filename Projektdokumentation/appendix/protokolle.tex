\addtocontents{toc}{\protect\contentsline{chapter}{Anhang}{}{}}

\chapter{Protokolle der Meetings}
\label{[chap:protokolle]}

\section{1. Treffen (02.10.15)}
\begin{itemize}
	\item bisherige Steuerung/Kontrolle von Smart-Home Geräten im Haus durch simple Zeit-/Anwesenheitssteuerung
	\item Probleme und Vorgaben:
	\begin{itemize}
		\item Beeinflussung des Systems von einer zweiten Person zur gleichen Zeit
		\item momentan existieren nur eine Person Steuerungskonzepte
		\item erarbeiten von Konzepten für mehrere Personen
		\item Wechselwirkung in der Beeinflussung des Systems z.B.: äußere Person beeinflusst Dinge die anwesende Person/en beeinflussen
		\item Besprochene Beispiele Jalousie-Schaltung, Feuermelder
		\item (Feedback von Systemänderungen an die Anwesenden Person) Feedback an Auslöser des Ereignisses
		\item Feedback über eingebaute Geräte Licht/Audio Effekte eventuell
		\item Interaktion durch Signale und vorhandene Schalter etc.
		\item Generationenkonflikt daher nicht Telefon/Smartphone fest anbinden
		\item geringe Kosten Massenmarkt/Low-Budget Bereich
		\item ZUno Projekt
	\end{itemize}
	
	\item Kreativität gefragt
	\item Ziel: Idealfall Szenarien mit praktischer Demonstration, sodass präsentierbar
	\item Analyse des Systems
	\item Z-Wave.eu \textrightarrow{ }Hohenstein
	\item Z-Wave.me \textrightarrow{ }Schweiz, Deutschland, Russland
	\item ZWay \textrightarrow{ }Softwarepaket/Server
	\item Hersteller übergreifende Kommunikation
	\item Simulation möglich
	\item mögliches Vorgehen: Recherche für Sensoren
	\item Android oder Webapp möglich (Teil von ZWay)
	\item Informationen durch die Foren
	\item Entwicklung zwei bis drei Szenarien das Mehr-Personen-Steuerung notwendig
	\item Vortrags-links werden bereitgestellt
	\item Organisatorisches
	\begin{itemize}
		\item Redmine Projekt
		\item Scrum
	\end{itemize}
\end{itemize}

\section{2. Treffen (09.10.15)}
\begin{itemize}
	\item Vortrag siehe Folien (Redmine)
	\begin{itemize}
		\item mehrere überlappende Netzwerke, wie ist das realisiert
		\item Gruppenarbeit Sensorrecherche bis nächste Woche
		\item Sensoren allgemein
		\item Aktivität anhand des Stromverbrauchs erkennen, Fensterkontakt
		\item weitere Technologien, die evtl. in Betracht gezogen werden können: Beacon, Z-Uno, RFID, NFC
		\item zusätzliche Informationsquellen einbinden, und so mehr über einen typischen Tagesablauf lernen, z.B. Terminkalender
		\item Vorstellung der Smart Home UI
		\item Frage ob Z-Uno gebraucht wird oder nicht
		\item was mit dessen Hilfe realisierbar ist
		\item Raspberry Pi bzw. Funkmodul (RaZberry besorgen)
		\item Z-Wave Programmierumgebung über Z-Wave oder JavaScript
	\end{itemize}
		
	\item Besprechung
	\begin{itemize}
		\item Sensorensuche
		\item Sensoren intelligent vernetzen
		\item Liste mit Sensoren (Z-Wave Europe)
		\item wie wird das mit einem Modell oder Raum
		\item wie können Erkenntnisse anschaulich vermittelt werden
		\item 2 Gruppen die jeweils 3 bis 5 Szenarien erarbeiten
		\item Gruppeneinteilung
	\end{itemize}
\end{itemize}
\section{3. Treffen (16.10.15)}

\begin{itemize}
	\item Vorstellung der in Gruppen erarbeiteten Szenarien
	\item Gruppe 1:
	\begin{itemize}
		\item Sicherheit im Heim durch Abschaltung von Gefahrenquellen (z.B. Haartrockner) bei Verlassen der Wohnung
		\item Gefahrenquellen (z.B. Ofen) deaktivieren, wenn keine Erwachsenen anwesend sind
	\end{itemize}
	\item Gruppe 2:
	\begin{itemize}
		\item Personenzähler je Raum anlegen, entsprechend der Personenanzahl unterschiedliche Aktionen ausführen
		\item Personenerkennung an der Tür durch außen und innen angebrachten Bewegungsmelder, evaluieren, wie zuverlässig diese Sensoren arbeiten und Grenzfälle testen (z.B. Begrüßung im Türbereich)
	\end{itemize}
	\item \gls{aal}
	\begin{itemize}
		\item erhebliche rechtliche Unsicherheiten
		\item technische Probleme, entsprechende Systeme können keine absolute Sicherheit bieten und nur unterstützend arbeiten
	\end{itemize}
	\item als weiterzuverfolgende Themen wurde identifiziert
	\begin{itemize}
		\item \gls{aal}
		\item intelligente Steuerung bei Verlassen der Wohnung
		\item Gefahrenquellen abstellen
		\item Smartphone als Präsenzsensor (Wifi, Funkzelle)
	\end{itemize}
	\item weiteres Vorgehen
	\begin{itemize}
		\item bereits erstellte Szenarien werden auf erforderliche Komponenten (Sensoren) überprüft
		\item Beschränkung auf Sensoren aus dem Z-Wave-Europe Produktkatalog
		\item Szenarien konkretisieren, mit hohem Detailgrad beschreiben (einen konkreten Ablauf)
		\item grafische Darstellung/Modellierung ausgewählter Szenarien
	\end{itemize}
\end{itemize}

\section{4. Treffen (19.10.15)}
\begin{itemize}
	\item Sensoren und Aktoren für Einkauf festgelegt (siehe Dokumente \textrightarrow{ }RelevanteSensoren.pdf)
	\item Organisatorisches
	\begin{itemize}
		\item ab diese Woche: Arbeitsergebnisse einen Tag vor einem Meeting, unter Dokumente entsprechend hochladen (bis 21:00 Uhr)
		\item die Modellierung der Szenarien ist mit dem Tool ArgoUML zu realisieren
		\item ArgoUML ist unter Dateien zu finden (einfach herunterladen, entpacken und starten)
		\item Dateiformat: strukturierte Textdatei (.txt ausreichend), um eine einheitliche Dokumentation zu ermöglichen alle bisherigen Szenarien sind nachzuarbeiten (siehe Ticket \#3961)
	\end{itemize}	
\end{itemize}

\section{5. Treffen (23.10.15)}
\begin{itemize}
	\item Präsentation:
	\begin{itemize}
		\item Organisatorisches
		\item Besprechung Handhabung Redmine
		\item Bilder als SVG
		\item für LaTex kompatible Dateien
		\item Stand der Dokumentation
		\item Gliederung der Dokumentation
		\item Arbeitsergebnisse von jedem selbständig
	\end{itemize}
	\item Vorstellung Szenarien
	\begin{itemize}
		\item Anzahl Personen im Raum (Philip Laube)
		\item Gefahrenquellen abschalten (Martin Petzold)
		\item Wohnung schließen (Simon Schwabe)
		\item Präsenzsensor (Patrick Hecker)
	\end{itemize}
	\item Modellierung
	\begin{itemize}
		\item Szenario oder Software lastig
		\item Ausbau Software und Benutzersicht der Modelle
		\item keine Anmerkung zu den Modellen
	\end{itemize}
	\item Besprochene Punkte
	\begin{itemize}
		\item als erstes Erkennung von Personen
		\item Feedback momentan LED an der Wand
		\item Alarm oder so
		\item Feedback innerhalb des Hauses
		\item Feedback in einem Beispielszenario allgemein
		\item Feedback muss angemessen sein und Frage wie realisiert ohne zu nerven
		\item Beispiel: abgedunkelter Raum Lampen bei Fußballtor ansteuern
		\item wie ist die Reaktion auf das Feedback
		\item z.B. Ereigniss Abstellung am Lichtschalter
		\item Z-Wave Taster als Beispiel Schalter mit mehr als 2 Zuständen der Geräte z.B. Schalter
		\item momentane Feedbackreaktion z.B. über App.
		\item Bedienelement und Smartphonebedienung \textrightarrow{ }Smartphone als Hardware-Ersatz für nicht vorhandene Hardwarekomponenten ("`als Krücke"')
		\item Vorgehensplan
		\item Modellüberarbeitung
		\item Z-Way Module planen/realisieren/testen
		\item Simulation mit Dummyelementen möglich
		\item Versuchsaufbau planen/realisieren/testen
		\item Simulation
		\item Sensoren evaluieren
	\end{itemize}
	\item Besprechung
	\begin{itemize}
		\item Bestellung muss realisiert werden
	\end{itemize}
	\item Vorgehensplan
	\begin{itemize}
		\item für das System Anwendung mehrere Module
		\item was sind Module \textrightarrow{ }nur Software
		\item Beispiel Module
		\item Komponenten für Szenarien überlegen
		\item Simulation lokal möglich
		\item bis nächste Woche Planung
		\item nächste Woche Architektur festlegen
	\end{itemize}
	\item Sprint 4
	\begin{itemize}
		\item Szenarien Modulsicht, Aufspaltung
		\item Modellüberarbeitung
		\item Zusammenarbeit der Szenarien festlegen
		\item Schnittstellen zwischen Modulen ist der interner Eventbus
		\item Nutzer Szenario Zuordnung
	\end{itemize}
	\item Zuordnung:
	\begin{itemize}
		\item Bewegungsmelder vor nach der Tür / Philip Laube
		\item Gefahrenquellen / Patrick Hecker, Martin Petzold, Tobias Weise
		\item Tür/Wohnung schließen / Alexander Keller, Simon Schwabe, Zarina Muratbekovna Omurova
	\end{itemize}
	\item Zerteilung der Szenarien
	\item Vorarbeit:
	\begin{itemize}
		\item Feedback, Versuchsaufbau (mobil)
		\item Erfahrungsbericht von Alexander Keller
		\item Sensorliste um Sirene erweitert? RGB-Lampe?
		\item Bestellung mit Kundenrabatt?
	\end{itemize}
\end{itemize}

\section{6. Treffen (30.10.15)}
\begin{itemize}
	\item Modelle der drei Gruppen besprechen
	\item Nachbearbeitung Modelle abgeschlossen
	\item Szenario Personenzählen
	\begin{itemize}
		\item Bewegungsmelderkonzept
		\item Personencounter
		\item Counter und Sensoren extra Module
	\end{itemize}

	\item Sensoren ungenau (Fibaro)
	\item Szenario Gefahrenquelle
	\begin{itemize}
		\item Sequenzdiagramm für Module vorgestellt
		\item Module selbst erklärt
		\item Modularisierung
		\item ungeklärt: Erkennung Erwachsener/Kind mit Sensoren schwierig!
		\item Trennung Personenanzahl von Information, ob Erwachsener/Kind
	\end{itemize}

	\item Szenario Wohnung verschließen
	\begin{itemize}
		\item Konfiguration
		\item In- und Output
		\item Reaktion des Alarmmodus auf dieses Modul
	\end{itemize}

	\item als Beispiel im Vergleich: Fibaro System
	\begin{itemize}
		\item Szenario Alarmmodul und Feedback
		\item Gliederung etc. siehe Präsentation
		\item Funktionalität
		\item Bose Sound System als Alarmsirene verwendet
		\item Lampen als Alarmgeber
		\item E-Mail Benachrichtigung
		\item Historisierungsidee und Benutzeroberfläche
		\item Fibaro System als Beispiel
		\item Editor eher ungeeignet, da nicht Massenmarktauglich
		\item Mehrbenutzer
		\item RFID bereits verworfen
		\item Smartphone Erkennung in Bearbeitung
		\item Personenerkennung wird entwickelt
	\end{itemize}

	\item ist eine eigene Oberfläche zu entwickeln?
	\begin{itemize}
		\item Test Vorstellung einiger Module
		\item Diskussion Oberflächenanpassung (alles neu oder nur Erweiterung?)
		\item niedrige Priorität, erst wenn Module funktional
	\end{itemize}
	
	\item Konzepte bis nächste Woche festlegen

	\begin{itemize}
		\item 05.11.2015 8:00 Uhr Einführung Modulentwicklung
		\item neue Gruppeneinteilung
	\end{itemize}

	\item Alternative Tickets freie Auswahl
	\begin{itemize}
		\item Versuchsaufbau
		\item Feedback
	\end{itemize}
	
\end{itemize}

\section{7. Treffen (05.11.15)}
\begin{itemize}
	\item Einführung in die Modul-Entwicklung
	\item Meeting des gesamten Teams
	\item Durchführung eines Workshops zur Entwicklung von Z-Way-Modulen
	\item Weiterentwicklung eines minimalen HelloWorld-Moduls
	\item dabei wurde bearbeitet
	\begin{itemize}
		\item Bereitstellen von Funktionen über die ZAutomation-API
		\item Senden und Empfangen von Events über den Eventbus
		\item Konfiguration von Modulen bei der Erzeugung
		\item Lesen aus der Konfigurationsdatei
		\item Arbeit mit Virtual Devices
		\item Verwenden der Metrics zur Speicherung von Daten in Modulen
		\item Einbinden externer JavaScript-Bibliotheken
	\end{itemize}
\end{itemize}

\section{8. Treffen (13.11.15)}
\begin{itemize}
	\item Test Einbindung bei Vorführung nicht erfolgreich, Bedienung schwierig
	\item Test Sensoren erfolgt
	\item Einbindung Raspberry Pi ins Netzwerk
	\item Test Verbindung Büro Testraum
	\item Stecker Vorführung und Inklusion
	\item Kurztest einiger Sensoren
	\item Diskussion über Verzögerungen bei Sensoren
	\item Test Steckdose, Türkontakt
	\item Sensor Emulation Vorschläge und Probleme
	\item eigene Module oder http Device
	\item Sprint 6:
	\begin{itemize}
			\item Sensoren Aktoren Evaluieren
			\item und Modulprogrammierung
			\item Gruppenweise Sensorprüfung
			\item Entwicklung der Module
	\end{itemize}	
\end{itemize}

\section{9. Treffen (20.11.15)}
\begin{itemize}
	\item Stand des Projekts momentan
	\item Verzögerung durch Sensorbestellung
	\item Planung ergänzen, Sensorankunft
	\item Besprechung, Zeitplanung, Zukunft
	\item Simon Schwabe befragt Leute einzeln für Dokumentation
	\item Besprechung Miniaturmodell
	\item Test Sensoren mit Miniaturen
	\item Module zur Personenzählung besprochen Übernahme von bereits vorhandener Implementierung
	\item Vorstellung Schnittstelle Personenzählung
	\item Vorstellung CO$_2$ Messungen, Theorie
	\item Probleme mit Zeitverzögerung der Bewegungsmelder
	\item Alarmmodul Vorführung
	\item Besprechung eigenes Netzwerk (WLAN) zur Steuerung
	\item Probe auf Laptop, wenn Raspberry Pi nicht funktioniert
	\item Feedbackmodul vorgestellt
	\item Aufgabenverteilung für nächste Woche
\end{itemize}

\section{10. Treffen (27.11.15)}

\section{11. Treffen (04.12.15)}
\begin{itemize}
	\item Projektstand
	\begin{itemize}
		\item Bericht Evaluierung
		\item Namenskonvention noch umsetzen
		\item Gefahrenquellen abstellen momentaner Stand
		\item Vorführung des Ablaufs
		\item Evaluation der Funktionsweise
	\end{itemize}

	\item Türschliessen Modul
	\begin{itemize}
		\item Vorführung Funktionserklärung
	\end{itemize}

	\item CO$_2$-Sensor
	\item Personenerkennung
	\begin{itemize}
		\item Unterscheidung hauptsächlich nach Kleinkind
		\item ältere Kinder nicht so wichtig (von denen kann Verhalten ähnlich Erwachsener erwartet werden)
		\item im Familienhaushalt:
		\begin{itemize}
			\item Bildung von Profilen
			\item Definition von Persona
			\item von Idealfall ausgehen, z.B. vorher definieren, wie viele Personen, Kinder, Erwachsene
		\end{itemize}
	\item allgemeingültige Lösungen sind nicht gefordert und nicht leistbar
		\begin{itemize}
			\item das erforderte zu detaillierte Informationen zu Personen
		\end{itemize}
	\item neue Aufgaben:
		\begin{itemize}
			\item Alexander Keller: Persona erstellen, analysieren, was bestimmte Personen an CO$_2$ verbrauchen	
		\end{itemize}
	\end{itemize}

	\item Tobias Weise: Versuchsaufbau
	\begin{itemize}
		\item Pavillon
		\item Stative, Raum, Feld
		\item Modellhaus
		\item Tests mit Modellhaus erfolgreich
		\item selbst bauen (beschlossen)
	\end{itemize}

	\item Aotec Multisensor
	\begin{itemize}
		\item Dokumentation
		\item EXPERT-UI nutzen
	\end{itemize}

	\item Schalter

	\begin{itemize}
		\item 4-Button Schalter
		\item kann als Controller dienen
	\end{itemize}

	\item Bemerkung: Zarina Muratbekovna Omurova war die letzten 2 Wochen krank
\end{itemize}

\section{12. Treffen (11.12.15)}
\subsubsection{Versuchsaufbau}

\begin{itemize}
	\item Tobias Weise hat einen Prototyp für einen Versuchsaufbau erstellt
	\item Pappkarton mit Legomännchen
	\item Problem dabei: Sensoren reagieren nicht zuverlässig auf Bewegungen der Figur
	\begin{itemize}
		\item Fibaro-Multisensor arbeitet am zuverlässigsten
	\end{itemize}
	\item Wie kann Puppenhaus umgebaut und genutzt werden? Überhaupt gewünscht?
	\begin{itemize}
		\item Puppenhaus eher mittel- und langfristig nutzen
		\item vorerst mit einem größeren Karton arbeiten
		\item reagieren Sensoren evtl. auf Wärme?
	\end{itemize}
\end{itemize}

\subsubsection{Persona \& Personenidentifikation (Alexander Keller)}

\begin{itemize}
	\item vier Personen, zwei Erwachsene, zwei Kinder
	\item ein Gespräch mit einem Arzt ergab:
	\begin{itemize}
		\item Personen können nicht anhand des CO$_2$-Ausstoßes identifiziert werden
		\item Mesomorpher Stoffwechsel, Ektomorpher Stoffwechsel, Endomorpher Stoffwechsel
		\item eine ruhende erwachsene Person stößt ähnlich viel CO$_2$ aus, wie ein Kind, welches körperlich aktiv ist
	\end{itemize}
	\item CO$_2$-Wert zur Erkennung, ob keine, eine oder mehrere Personen anwesend sind
	\item CO$_2$-Wert kann nicht wirklich genutzt werden
	\item weiterhin auf Größenmessung an der Tür konzentrieren
\end{itemize}

\subsubsection{Feedback-Mechanismus}

\begin{itemize}
	\item wird in letzter Instanz genutzt, um anwesende Personen entscheiden zu lassen, ob Einschätzungen/Entscheidungen des System korrekt waren
\end{itemize}

\subsubsection{Weiteres Vorgehen}

\begin{itemize}
	\item planen der Präsentation
	\begin{itemize}
		\item roten Faden erarbeiten
		\item ca. 30 - 40 min
		\item entweder in Prüfungszeit, wenn nicht bis dahin, dann in den ersten Wochen des nächsten Semesters
		\item Festlegung in der ersten Januarwoche
	\end{itemize}
	\item weiterhin Feedbackmöglichkeiten erarbeiten
	\item Tobias Weise arbeitet am Versuchsaufbau weiter
\end{itemize}

\subsubsection{Dokumentation}

\begin{itemize}
	\item Simon Schwabe: Vorbereitung für Dokumentation (Was ist vorhanden, was muss noch ergänzt werden?)
	\item Gliederung mit Prof. Golubski absprechen
	\item Philip Laube: 1. Einleitung und 2. Projektziel
\end{itemize}


\section{13. Treffen (18.12.15)}
\subsubsection{Organisatorisches}
\begin{itemize}
	\item zusätzliches Treffen am Montag, 04.01.2016 ca. 16:00 Uhr Skype
	\item Präsentation:
	\begin{itemize}
		\item Projektleiter als Ersatz, wenn Präsentator ausfällt
		\item Präsentation in zweiter Prüfungswoche (evtl. 05.02.16)
		\item Bericht bis 22.01.16 (letzte Vorlesungswoche) abgeben
	\end{itemize}
	\item für 15/16.01.16 ab 11:20 Uhr Prof. Pätz einladen, Präsentation der Ergebnisse
	\item Präsentation, von der Teile weiterverwendet werden können, erarbeiten
\end{itemize}
\subsubsection{Feedback-Mechanismus}
\begin{itemize}
	\item akustische Wahrnehmung
	\begin{itemize}
		\item Sprachsteuerung: Nutzer steuert System über Sprachbefehle
		\item aktuelle Forschung:
		\item SCARS Speech Control an dAudio Response System
		\item LOGOS
	\end{itemize}
	\item visuelle Wahrnehmung
	\begin{itemize}
		\item verschiedene Farbkodierungen für unterschiedliche anstehende Aktionen, z.B. grünes Licht \textrightarrow{ }Abschaltung der Heizung
	\end{itemize}
	\item Gestensteuerung
	\begin{itemize}
		\item Gestenerkennung mit Infrarotkameras, Steuerung durch Gesten
	\end{itemize}
	\item weitere Überlegung
	\begin{itemize}
		\item was ist mit Z-Wave möglich
		\item welche Kosten entstehen durch Systeme (z.B. Kinect)
		\item wie alltagstauglich?
	\end{itemize}
\end{itemize}

\subsubsection{Versuchsaufbau}
\begin{itemize}
	\item ähnlich wie letzter Aufbau, allerdings etwas größer
\end{itemize}

\subsubsection{Sensorevaluation}
\begin{itemize}
	\item Bewegungsmeder zur Feststellung der Größe
	\begin{itemize}
		\item funktioniert technisch, nicht praxistauglich (siehe Dokumentation)
	\end{itemize}CO$_2$-Messung
	\begin{itemize}
		\item unter Idealbedingungen funktioniert das Modul, wie es soll
		\item Modul ist träge (\textrightarrow{ }CO$_2$-Änderung)
	\end{itemize}
\end{itemize}

\subsubsection{Sprint 11 (Vorlesungsfreie Zeit)}
\begin{itemize}
	\item Dokumentation
	\item Arbeitsergebnisse zusammenfassen
	\item Szenarien mit Modulen in Verbindung bringen und Versuchsaufbau für Szenarien
	\item Gefahrenquellen abstellen, Aufbau, Dokumentation: Philip Laube
	\item Tür schließen, Aufbau, Dokumentation Simon Schwabe
	\item Projektverlauf, Vorgehensplan: Patrick Hecker
	\item Reviewplan für Dokumentation und Dokumente
	\begin{itemize}
		\item wer liest wann was?
		\item Anmerkungen geben
		\item grobe Gliederungspunkte werden von Autor + einem Reviewer überprüft
		\item Absprache der beiden \textrightarrow{ }Änderungen in Dokumentation übernehmen
		\item auch Formulierungen überarbeiten
	\end{itemize}
	\item Fazit (auch: Reflexion der eigenen Arbeit): später
\end{itemize}

\subsubsection{Anmerkung}
\begin{itemize}
	\item Alexander Keller war heute nicht anwesend
\end{itemize}


\section{14. Treffen (04.01.16)}
\begin{itemize}
	\item Vorstellung der Präsentation (aktueller Stand)
	\begin{itemize}
		\item Ideen an Alexander Keller senden (aktueller Stand als PDF verfügbar)
		\item Vorstellung der Szenarien zur Präsentation (1. Live, 2. Simulation, 3. Video)
	\end{itemize}
	\item Entwicklung der Module abgeschlossen
	\begin{itemize}
		\item Versuchsaufbau folgt im Laufe der Woche
	\end{itemize}
	\item Zwischenergebnisse Dokumentation (zum Review bereit) in Dokumente \textrightarrow{ }aktueller Stand der Projekt-Dokumentation ablegen und in der Google Tabelle verlinken
\end{itemize}


\section{15. Treffen (08.01.16)}
\begin{itemize}
	\item Organisation:
	\begin{itemize}
		\item Donnerstag, 14.01.16: Treffen mit Prof. Pätz, Präsentation bisheriger Ergebnisse
		\item Abschlussmeeting: 15.01.16
		\item Dokumentation: bis 22.01.16
	\end{itemize}

	\item Aktueller Stand des Versuchsaufbau
	\begin{itemize}
		\item Aufbau und Funktionsweise
		\item noch "tapezieren", z.B. mit Sprühfolie oder Klebefolie bekleben
	\end{itemize}

	\item Feedback-Mechanismus
	\begin{itemize}
		\item IVEE: cloudbasierte Sprachsteuerung
		\item Amazon Echo: Sprachsteuerung direkt im Raum, auch "Fernbedienung" verfügbar
		\item Homey: Interface um Sprachbefehle entgegenzunehmen
		\begin{itemize}
			\item arbeitet mit verschiedenen Protokollen, z.B. Z-Wave, Infrarot, ZigBee
		\end{itemize}
		\item Enblink: USB-Stick, z.B. für Google-TV
		\item Z-Wave.Me Fernbedienung zur Steuerung von Z-Wave Geräten
		\item OneCue: funktioniert ähnlich wie Kinect-Kamera
		\item Reemo: Gestensteuerung am Handgelenk
		\item HomeKit: Apple-Framework zur Sprachsteuerung
	\end{itemize}

	\item Aktueller Stand der Präsentation
	\begin{itemize}
		\item wie ist Organisation, wie lief Projektarbeit
		\begin{itemize}
			\item gerne kritische Betrachtung, was gelernt, wie war Dynamik, wie hat sich Gruppenarbeit entwickelt
			\item Retrospektive auf Projektarbeit und Projektablauf
		\end{itemize}
		\item Präsentation evtl. früher, um Interesse zu wecken, z.B. nach Szenarien
		\begin{itemize}
			\item Entwicklung erst danach
			\item noch eine Präsentation/Demonstration ausarbeiten
		\end{itemize}
		\item Begrifflichkeiten bei der Kategorisierung von Smart Homes noch überprüfen
		\item Schriftgrößen überprüfen
		\item Feedback-Ausarbeitungen übernehmen
		\item Modulentwickler liefern noch Informationen für die Präsentation und stehen dann auch für Rückfragen bereit
		\begin{itemize}
			\item dazu noch einen Überblick erstellen, welche Informationen noch benötigt werden
		\end{itemize}
		\item Storyboard/Timeline erarbeiten, wie lange auf welcher Folie? Wo liegen Schwerpunkte?
		\item Probleme direkt bei Entwicklung beschreiben, später nur zusammenfassend
		\item Präsentation der Module anhand des Moduls
		\begin{itemize}
			\item 	evtl. mit Video, vom Versuchsaufbau
		\end{itemize}
	\end{itemize}

	\item Anmerkung
	\begin{itemize}
		\item Martin Petzold ist heute krank
	\end{itemize}
\end{itemize}

\section{15. Treffen (08.01.16)}