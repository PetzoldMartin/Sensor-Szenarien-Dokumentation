\addtocontents{toc}{\protect\contentsline{chapter}{Anhang}{}{}}

\chapter{Protokolle der Meetings}
\label{[chap:protokolle]}

\section{1. Treffen (02.10.15)}
\begin{itemize}
	\item bisherige Steuerung/Kontrolle von Smart-Home Geräten im Haus durch simple Zeit-/Anwesenheitssteuerung
	\item Probleme und Vorgaben:
	\begin{itemize}
		\item Beeinflussung des Systems von einer zweiten Person zur gleichen Zeit
		\item momentan existieren nur eine Person Steuerungskonzepte
		\item erarbeiten von Konzepten für mehrere Personen
		\item Wechselwirkung in der Beeinflussung des Systems z.B.: äußere Person beeinflusst Dinge die anwesende Person/en beeinflussen
		\item Besprochene Beispiele Jalousie-Schaltung, Feuermelder
		\item (Feedback von Systemänderungen an die Anwesenden Person) Feedback an Auslöser des Ereignisses
		\item Feedback über eingebaute Geräte Licht/Audio Effekte eventuell
		\item Interaktion durch Signale und vorhandene Schalter etc.
		\item Generationenkonflikt daher nicht Telefon/Smartphone fest anbinden
		\item geringe Kosten Massenmarkt/Low-Budget Bereich
		\item ZUno Projekt
	\end{itemize}
	
	\item Kreativität gefragt
	\item Ziel: Idealfall Szenarien mit praktischer Demonstration, sodass präsentierbar
	\item Analyse des Systems
	\item Z-Wave.eu \textrightarrow Hohenstein
	\item Z-Wave.me \textrightarrow Schweiz, Deutschland, Russland
	\item ZWay \textrightarrow Softwarepaket/Server
	\item Hersteller übergreifende Kommunikation
	\item Simulation möglich
	\item mögliches Vorgehen: Recherche für Sensoren
	\item Android oder Webapp möglich (Teil von ZWay)
	\item Informationen durch die Foren
	\item Entwicklung zwei bis drei Szenarien das Mehr-Personen-Steuerung notwendig
	\item Vortrags-links werden bereitgestellt
	\item Organisatorisches
	\begin{itemize}
		\item Redmine Projekt
		\item Scrum
	\end{itemize}
\end{itemize}

\section{2. Treffen (09.10.15)}
\begin{itemize}
	\item Vortrag siehe Folien (Redmine)
	\begin{itemize}
		\item mehrere überlappende Netzwerke, wie ist das realisiert
		\item Gruppenarbeit Sensorrecherche bis nächste Woche
		\item Sensoren allgemein
		\item Aktivität anhand des Stromverbrauchs erkennen, Fensterkontakt
		\item weitere Technologien, die evtl. in Betracht gezogen werden können: Beacon, Z-Uno, RFID, NFC
		\item zusätzliche Informationsquellen einbinden, und so mehr über einen typischen Tagesablauf lernen, z.B. Terminkalender
		\item Vorstellung der Smart Home UI
		\item Frage ob Z-Uno gebraucht wird oder nicht
		\item was mit dessen Hilfe realisierbar ist
		\item Raspberry Pi bzw. Funkmodul (RaZberry besorgen)
		\item Z-Wave Programmierumgebung über Z-Wave oder JavaScript
	\end{itemize}
		
	\item Besprechung
	\begin{itemize}
		\item Sensorensuche
		\item Sensoren intelligent vernetzen
		\item Liste mit Sensoren (Z-Wave Europe)
		\item wie wird das mit einem Modell oder Raum
		\item wie können Erkenntnisse anschaulich vermittelt werden
		\item 2 Gruppen die jeweils 3 bis 5 Szenarien erarbeiten
		\item Gruppeneinteilung
	\end{itemize}
\end{itemize}
\section{3. Treffen (16.10.15)}

\begin{itemize}
	\item Vorstellung der in Gruppen erarbeiteten Szenarien
	\item Gruppe 1:
	\begin{itemize}
		\item Sicherheit im Heim durch Abschaltung von Gefahrenquellen (z.B. Haartrockner) bei Verlassen der Wohnung
		\item Gefahrenquellen (z.B. Ofen) deaktivieren, wenn keine Erwachsenen anwesend sind
	\end{itemize}
	\item Gruppe 2:
	\begin{itemize}
		\item Personenzähler je Raum anlegen, entsprechend der Personenanzahl unterschiedliche Aktionen ausführen
		\item Personenerkennung an der Tür durch außen und innen angebrachten Bewegungsmelder, evaluieren, wie zuverlässig diese Sensoren arbeiten und Grenzfälle testen (z.B. Begrüßung im Türbereich)
	\end{itemize}
	\item \gls{aal}
	\begin{itemize}
		\item erhebliche rechtliche Unsicherheiten
		\item technische Probleme, entsprechende Systeme können keine absolute Sicherheit bieten und nur unterstützend arbeiten
	\end{itemize}
	\item als weiterzuverfolgende Themen wurde identifiziert
	\begin{itemize}
		\item \gls{aal}
		\item intelligente Steuerung bei Verlassen der Wohnung
		\item Gefahrenquellen abstellen
		\item Smartphone als Präsenzsensor (Wifi, Funkzelle)
	\end{itemize}
	\item weiteres Vorgehen
	\begin{itemize}
		\item bereits erstellte Szenarien werden auf erforderliche Komponenten (Sensoren) überprüft
		\item Beschränkung auf Sensoren aus dem Z-Wave-Europe Produktkatalog
		\item Szenarien konkretisieren, mit hohem Detailgrad beschreiben (einen konkreten Ablauf)
		\item grafische Darstellung/Modellierung ausgewählter Szenarien
	\end{itemize}
\end{itemize}

\section{4. Treffen (19.10.15)}
\begin{itemize}
	\item Sensoren und Aktoren für Einkauf festgelegt (siehe Dokumente \textrightarrow RelevanteSensoren.pdf)
	\item Organisatorisches
	\begin{itemize}
		\item ab diese Woche: Arbeitsergebnisse einen Tag vor einem Meeting, unter Dokumente entsprechend hochladen (bis 21:00 Uhr)
		\item die Modellierung der Szenarien ist mit dem Tool ArgoUML zu realisieren
		\item ArgoUML ist unter Dateien zu finden (einfach herunterladen, entpacken und starten)
		\item Dateiformat: strukturierte Textdatei (.txt ausreichend), um eine einheitliche Dokumentation zu ermöglichen alle bisherigen Szenarien sind nachzuarbeiten (siehe Ticket \#3961)
	\end{itemize}	
\end{itemize}

\section{5. Treffen (23.10.15)}
\begin{itemize}
	\item Präsentation:
	\begin{itemize}
		\item Organisatorisches
		\item Besprechung Handhabung Redmine
		\item Bilder als SVG
		\item für LaTex kompatible Dateien
		\item Stand der Dokumentation
		\item Gliederung der Dokumentation
		\item Arbeitsergebnisse von jedem selbständig
	\end{itemize}
	\item Vorstellung Szenarien
	\begin{itemize}
		\item Anzahl Personen im Raum (Laube)
		\item Gefahrenquellen abschalten (Petzold)
		\item Wohnung schließen (Schwabe)
		\item Präsenzsensor (Hecker)
	\end{itemize}
	\item Modellierung
	\begin{itemize}
		\item Szenario oder Software lastig
		\item Ausbau Software und Benutzersicht der Modelle
		\item keine Anmerkung zu den Modellen
	\end{itemize}
	\item Besprochene Punkte
	\begin{itemize}
		\item als erstes Erkennung von Personen
		\item Feedback momentan LED an der Wand
		\item Alarm oder so
		\item Feedback innerhalb des Hauses
		\item Feedback in einem Beispielszenario allgemein
		\item Feedback muss angemessen sein und Frage wie realisiert ohne zu nerven
		\item Beispiel: abgedunkelter Raum Lampen bei Fußballtor ansteuern
		\item wie ist die Reaktion auf das Feedback
		\item z.B. Ereigniss Abstellung am Lichtschalter
		\item Z-Wave Taster als Beispiel Schalter mit mehr als 2 Zuständen der Geräte z.B. Schalter
		\item momentane Feedbackreaktion z.B. über App.
		\item Bedienelement und Smartphonebedienung -> Smartphone als Hardware-Ersatz für nicht vorhandene Hardwarekomponenten ("`als Krücke"')
		\item Vorgehensplan
		\item Modellüberarbeitung
		\item Z-Way Module planen/realisieren/testen
		\item Simulation mit Dummyelementen möglich
		\item Versuchsaufbau planen/realisieren/testen
		\item Simulation
		\item Sensoren evaluieren
	\end{itemize}
	\item Besprechung
	\begin{itemize}
		\item Bestellung muss realisiert werden
	\end{itemize}
	\item Vorgehensplan
	\begin{itemize}
		\item für das System Anwendung mehrere Module
		\item was sind Module \textrightarrow nur Software
		\item Beispiel Module
		\item Komponenten für Szenarien überlegen
		\item Simulation lokal möglich
		\item bis nächste Woche Planung
		\item nächste Woche Architektur festlegen
	\end{itemize}
	\item Sprint 4
	\begin{itemize}
		\item Szenarien Modulsicht, Aufspaltung
		\item Modellüberarbeitung
		\item Zusammenarbeit der Szenarien festlegen
		\item Schnittstellen zwischen Modulen ist der interner Eventbus
		\item Nutzer Szenario Zuordnung
	\end{itemize}
	\item Zuordnung:
	\begin{itemize}
		\item Bewegungsmelder vor nach der Tür / Philip Laube
		\item Gefahrenquellen / Patrick Hecker, Martin Petzold, Tobias Weise
		\item Tür/Wohnung schließen / Alexander Keller, Simon Schwabe, Zarina Omurova
	\end{itemize}
	\item Zerteilung der Szenarien
	\item Vorarbeit:
	\begin{itemize}
		\item Feedback, Versuchsaufbau (mobil)
		\item Erfahrungsbericht von Alex
		\item Sensorliste um Sirene erweitert? RGB-Lampe?
		\item Bestellung mit Kundenrabatt?
	\end{itemize}
\end{itemize}

\section{6. Treffen (30.10.15)}
\begin{itemize}
	\item Modelle der drei Gruppen besprechen
	\item Nachbearbeitung Modelle abgeschlossen
	\item Szenario Personenzählen
	\begin{itemize}
		\item Bewegungsmelderkonzept
		\item Personencounter
		\item Counter und Sensoren extra Module
	\end{itemize}

	\item Sensoren ungenau (Fibaro)
	\item Szenario Gefahrenquelle
	\begin{itemize}
		\item Sequenzdiagramm für Module vorgestellt
		\item Module selbst erklärt
		\item Modularisierung
		\item ungeklärt: Erkennung Erwachsener/Kind mit Sensoren schwierig!
		\item Trennung Personenanzahl von Information, ob Erwachsener/Kind
	\end{itemize}

	\item Szenario Wohnung verschließen
	\begin{itemize}
		\item Konfiguration
		\item In- und Output
		\item Reaktion des Alarmmodus auf dieses Modul
	\end{itemize}

	\item als Beispiel im Vergleich: Fibaro System
	\begin{itemize}
		\item Szenario Alarmmodul und Feedback
		\item Gliederung etc. siehe Präsentation
		\item Funktionalität
		\item Bose Sound System als Alarmsirene verwendet
		\item Lampen als Alarmgeber
		\item E-Mail Benachrichtigung
		\item Historisierungsidee und Benutzeroberfläche
		\item Fibaro System als Beispiel
		\item Editor eher ungeeignet, da nicht Massenmarktauglich
		\item Mehrbenutzer
		\item RFID bereits verworfen
		\item Smartphone Erkennung in Bearbeitung
		\item Personenerkennung wird entwickelt
	\end{itemize}

	\item ist eine eigene Oberfläche zu entwickeln?
	\begin{itemize}
		\item Test Vorstellung einiger Module
		\item Diskussion Oberflächenanpassung (alles neu oder nur Erweiterung?)
		\item niedrige Priorität, erst wenn Module funktional
	\end{itemize}
	
	\item Konzepte bis nächste Woche festlegen

	\begin{itemize}
		\item 05.11.2015 8:00 Uhr Einführung Modulentwicklung
		\item neue Gruppeneinteilung
	\end{itemize}

	\item Alternative Tickets freie Auswahl
	\begin{itemize}
		\item Versuchsaufbau
		\item Feedback
	\end{itemize}
	
\end{itemize}

\section{7. Treffen (05.11.15)}
\begin{itemize}
	\item Einführung in die Modul-Entwicklung
	\item Meeting des gesamten Teams
	\item Durchführung eines Workshops zur Entwicklung von Z-Way-Modulen
	\item Weiterentwicklung eines minimalen HelloWorld-Moduls
	\item dabei wurde bearbeitet
	\begin{itemize}
		\item Bereitstellen von Funktionen über die ZAutomation-API
		\item Senden und Empfangen von Events über den Eventbus
		\item Konfiguration von Modulen bei der Erzeugung
		\item Lesen aus der Konfigurationsdatei
		\item Arbeit mit Virtual Devices
		\item Verwenden der Metrics zur Speicherung von Daten in Modulen
		\item Einbinden externer JavaScript-Bibliotheken
	\end{itemize}
\end{itemize}

\section{8. Treffen (13.11.15)}
\begin{itemize}
	\item Test Einbindung bei Vorführung nicht erfolgreich, Bedienung schwierig
	\item Test Sensoren erfolgt
	\item Einbindung Raspberry Pi ins Netzwerk
	\item Test Verbindung Büro Testraum
	\item Stecker Vorführung und Inklusion
	\item Kurztest einiger Sensoren
	\item Diskussion über Verzögerungen bei Sensoren
	\item Test Steckdose, Türkontakt
	\item Sensor Emulation Vorschläge und Probleme
	\item eigene Module oder http Device
	\item Sprint 6:
\begin{itemize}
		\item Sensoren Aktoren Evaluieren
		\item und Modulprogrammierung
		\item Gruppenweise Sensorprüfung
		\item Entwicklung der Module
\end{itemize}
	
\end{itemize}