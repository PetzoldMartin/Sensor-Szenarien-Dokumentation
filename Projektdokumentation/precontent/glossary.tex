% \newglossaryentry{⟨label⟩}
% {
% name={⟨abbrv⟩},
% description={⟨long)}
% }

%A
%B
%C
%D
%E
%F
%G
%H
%I
\newglossaryentry{iftttg}
{
	name={IFTTT},
	description={IFTTT ermöglicht Benutzern sogenannte \glqq Rezepte\grqq{} nach dem Motto \glqq If this then that\grqq{} (\glqq Wenn dies dann das\grqq) zu erstellen. Der \glqq this\grqq -Teil eines Rezepts wird auch \glqq Trigger\grqq{} genannt. Ein Trigger kann zum Beispiel sein: \glqq Du wurdest auf einem Foto bei Facebook markiert\grqq{} oder \glqq Ich habe bei Foursquare eingecheckt\grqq . Der \glqq that\grqq -Teil eines Rezepts wird Aktion genannt. Eine Aktion kann zum Beispiel sein: \glqq Sende mir eine Textnachricht\grqq{} oder \glqq Schalte das Licht ein\grqq . Die Kombination aus Trigger und Aktion wird Rezept genannt. IFTTT bietet Trigger und Aktionen für mehr als 100 Dienste wie Twitter, Foursquare, Flickr, Facebook.}
}
%J
%K
%L
%N
\newglossaryentry{nfcg}
{
	name={NFC},
	description={NFC ist eine drahtlose Kommunikationstechnologie mit limitierter Datenrate und Reichweite und basiert auf \gls{rfid}. NFC findet z.B. in Mobiltelefonen zum bargeldlosen bezahlen Anwendung}
}
%M
\newglossaryentry{mkqg}
{
	name={MKQ},
	description={Die Methode der kleinsten Quadrate ist ein mathematisches Standardverfahren zur Ausgleichsrechnung}
}
%O
%P

%Q
%R
%S

%T
%U

%V
%W

%X
%Y
%Z
\newglossaryentry{zigbeeg}
{
	name={ZigBee},
	description={Von der ZigBee-Allianz entwickelte Spezifikation für drahtlose Netze, welcher besonders im Hausautomatisierungsbereich verbreitet ist}
}

\newglossaryentry{zwaveg}
{
	name={Z-Wave},
	description={Von Sigma Designs entwickelter drahtloser Kommunikationsstandard für Hausautomatisierung. Besonderheit ist dabei, dass auch die Anwendungsebene spezifiziert ist und Geräte so herstellerunabhängig kommunizieren können}
}

\newglossaryentry{zwayg}
{
	name={Z-Way},
	description={Eine vom Unternehmen Z-Wave.Me entwickelte Serveranwendung zur Nutzung und Verwaltung von Z-Wave-Netzen}
}